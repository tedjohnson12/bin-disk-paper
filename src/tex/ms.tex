% Define document class
\documentclass[twocolumn]{aastex631}
\usepackage{showyourwork}

\newcommand\ghurl[0]{\url{https://github.com/tedjohnson12/bin-disk-paper}}

% Begin!
\begin{document}

% Title
\title{Probabilites of polar aligned disks in binary star systems}

% Author list
\author{Ted Johnson}
\affiliation{University of Nevada, Las Vegas}
\author{Rebecca Martin}
\affiliation{University of Nevada, Las Vegas}
\author{Stephen Lepp}
\affiliation{University of Nevada, Las Vegas}

% Abstract with filler text
\begin{abstract}
    Inclined circumbinary disks settle to one of three configurations: 1) prograde coplanar,
    2) retrograde coplanar, and 3) polar aligned. The preferred alignment can be found analytically
    in the case of a massless disk, however disk-star interactions make the general case much more challenging.
    We employ the N-body code REBOUND to investigate the dynamics of these 3-body systems and use monte-carlo methods
    to quantify the probabilities of the various configurations in a given system.
\end{abstract}

% Main body with filler text
\section{Introduction}
\label{sec:intro}

[\textcolor{red}{Why?}]

Circumbinary planets -- those orbiting two stars -- have been mostly found to orbit in the same plane as
their hosts \citep{2011Sci...333.1602D,2012ApJ...758...87O,2012Natur.481..475W}. However, \citet{2023ApJ...957L..28M}
found evidence for a circumbinary planet that orbits nearly perpendicular to the binary orbinal plane.


\section{Acknowledgements}
\label{sec:ack}

This manuscript was prepared using the open-science software \href{https://show-your.work/en/latest/intro/}{\showyourwork} \citep{Luger2021}, making the article completely
reproducible. The source code to compile this document and create the figures is available on GitHub\footnote{\ghurl}.

[\textcolor{red}{REBOUND}]

[\textcolor{red}{Cherry Creek}]

\bibliography{bib}

\end{document}
