% Define document class
\documentclass[twocolumn]{aastex631}
\usepackage{showyourwork}
\usepackage{amsmath}
\usepackage{bm}

\newcommand\ghurl[0]{\url{https://github.com/tedjohnson12/bin-disk-paper}}
\newcommand\codeurl[0]{\url{https://github.com/tedjohnson12/misaligned_cb_disk}}
\newcommand\getref[0]{(\textcolor{red}{ref?})}
\newcommand\question[1]{[\textcolor{red}{#1}]}

\newcommand{\RGM}[1]{\textcolor{cyan}{#1}}

\DeclareMathOperator{\arctantwo}{arctan2}

% Begin!
\begin{document}

% Title
\title{Probabilities of polar aligned disks in binary star systems}

% Author list
\author{Ted Johnson}
%\affiliation{Department of Physics and Astronomy, University of Nevada, Las Vegas, 4505 South Maryland Parkway, Las Vegas, NV 89154, USA}
\author{Rebecca G. Martin}
%\affiliation{Department of Physics and Astronomy, University of Nevada, Las Vegas, 4505 South Maryland Parkway, Las Vegas, NV 89154, USA}
\author{Stephen Lepp}
\affiliation{Nevada Center for Astrophysics, University of Nevada, Las Vegas, 4505 South Maryland Parkway, Las Vegas, NV 89154, USA}
\affiliation{Department of Physics and Astronomy, University of Nevada, Las Vegas, 4505 South Maryland Parkway, Las Vegas, NV 89154, USA}

% Abstract with filler text
\begin{abstract}
    Circumbinary gas disks that are misaligned to the binary orbital plane evolve toward either a coplanar or a polar-aligned configuration with respect to the binary host.
    The preferred alignment can be found analytically
    in the limit of a massless disk, however disk-binary interactions make the massive disk case much more challenging.
    We employ the N-body code {\sc rebound} to investigate the dynamics of  3-body systems and use Monte Carlo methods
    to quantify the probabilities of each alignment for an initially randomly orientated disk around a given system.
\end{abstract}

\section{Introduction}
\label{sec:intro}



Observed circumbinary planets -- those orbiting around two stars -- have been mostly found to orbit in the same plane as
their hosts \citep{doyle2011,orosz2012,welsh2012}. However, this is likely a result of selection effects \RGM{(citations by David Martin..)}.  Misaligned disks around eccentric binaries evolve \RGM{either to coplanar (citations)} to a polar state  \citep{aly2015,martin2017}. A polar-aligned debris disk has been found in the 99 Her system
\citep{kennedy2012} and polar-aligned gas disks have been found in the HD 98800 \citep{kennedy2019} and V773 Tau systems \citep{kenworthy2022}.
AC Her, a post-AGB system with a polar-inclined
disk, may host the first circumbinary planet found in a polar orbit \citep{hillen2015,anugu2023,martin2023}.


The alignment of circumbinary planets, and the alignment of the disks from which they form, are determined by their interaction history with their
host stars. \RGM{For a massless} circumbinary disk, the type of nodal precession can be split into two categories:

\begin{enumerate}
    \item {\bf Precession about the angular momentum vector of the binary.} In the case where the planet/disk
        is slightly misalignend, its own angular momentum vector precesses about that of the binary \citep[e.g.,][]{bate2000,lubow2000}.
        This is characterized by near constant
        inclination and oscillation of the longitude of the ascending node. Included in this case is the subclassification
        of retrograde orbits (those whose angular momentum vectors are misaligend by $\sim 180 \deg$) and the two
        trivial cases of perfectly aligned/misaligend orbits which do not precess. \label{enum:precession}
    \item {\bf Libration about the eccentricity vector of the binary.} In the case where misalignment is high (i.e. near
    perpendicular) the angular momentum vector of the planet/disk precesses about the eccentricity vector of the binary 
    \citep[e.g.,][]{verrier2009,farago2010,doolin2011}. This is called a librating orbit. In this case the mutual inclination of the disk oscillates about the polar configuration.
    \label{enum:libration}
\end{enumerate}

\RGM{However, for a massive disk, there are additional types of nodal precesion.... (describe) \citep{abod2022}.}

Planets form with the orbital characteristics given by their progenitor disks \citep[e.g.,][]{childs2021}, so it is sufficient in this work to
discuss disk dynamics only.
Differential precession between adjacent annular regions in a disk cause dissipation, ultimately dampening precession or libration
so that the disk reaches a steady state on its viscous timescale \citep[see also \citet{nixon2011,foucart2013,foucart2014}]{bate2000}.
Therefore, it is appropriate to map a disk's
initial interactions with a binary (prograde precession, libration, retrograde precession) to the disk's final configuration
(prograde coplanar, circumpolar, retrograde coplanar, respectively).

In the case of a massless disk, the type of interaction can be determined analytically from just the binary eccentricity and the
direction of the disk's specific angular momentum vector
\citep[parameterized by the disk inclination and its longitude of the ascending node][]{zanazzi2018}.
Given priors on these quantities we can predict the observed frequencies of circumbinary disks (and planets) in the polar configuration.
\citet{ceppi2024} do just this -- finding the fractions of binaries with polar disks and the mean eccentricities of those binaries
as a function of two free parameters.

In this work we will study the more general case of a massive disk. The resulting polar frequency function only contains one
additional parameter, but the orbital parameters of the binary can now change due to interactions with the disk. The additional
complications of this case require the application of numerical simulations; we use the N-body code {\sc rebound} \citep{rebound} to simulate the
dynamics of a three-body system, with a massive point particle used as a stand-in for the disk. \citet{abod2022} found that this 3-body
setup is a valid description of a disk with angular momentum equal to that of the third body in the limit that the disk mass does not change
(i.e. no accretion). This setup also cannot account for warping effects and disk breakage as we must treat the disk as a solid body. 
We describe this setup in detail in Section \ref{sec:methods}

\question{describe the regime we are working in}


\section{Methods}
\label{sec:methods}
We use the N-body code {\sc rebound} \citep{rebound} to simulate a 3-body system. The code used to run these simulations is publically
available\footnote{\codeurl}.

\subsection{Simulation Setup}
\label{subsec:setup}

\begin{table*}
    \begin{tabular}{cccp{0.5\textwidth}}
        \hline
        name & symbol & type & description \\
        \hline
        \texttt{mass\_binary} & $M_b$ & \texttt{float} & The total mass of the binary in solar masses. \\
        \texttt{mass\_fraction} & $f_b$ & \texttt{float} & The ratio between the secondary mass and the binary mass $M_2/M_b$. \\
        \texttt{semimajor\_axis\_binary} & $a_b$ & \texttt{float} & The semimajor axis of the binary orbit. \\
        \texttt{eccentricity\_binary} & $e_b$ & \texttt{float} & The eccentricity of the binary orbit. \\
        \texttt{mass\_planet} & $m_p$ & \texttt{float} & The mass of the third body. \\
        \texttt{semimajor\_axis\_planet} & $a_p$ & \texttt{float} & The semimajor of the third body orbiting the binary center of mass. \\
        \texttt{inclination\_planet} & $i$ & \texttt{float} & The mutual inclination of the binary and the third body. \\
        \texttt{lon\_ascending\_node\_planet} & $\Omega$ & \texttt{float} & The longitude of the ascending node of the third body. \\
        \texttt{true\_anomaly\_planet} & $\nu$ & \texttt{float} & The true anomaly of the third body. \\
        \texttt{gr} & - & \texttt{bool} & Whether to include effects from general relativity. \\
        \texttt{eccentricity\_planet} & $e_p$ & \texttt{float} & The eccentricity of the third body's orbit. \\
        \texttt{arg\_pariapsis\_planet} & $\omega$ & \texttt{float} & The argument of pariapsis of the third body's orbit. \\
        \hline
    \end{tabular}
    \caption{Simulation parameters}
    \label{tab:params}
\end{table*}

We use {\sc rebound}'s Python interface to initialize our simulation. Table \ref{tab:params} describes the simulation's
input parameters. We integrate the system using the IAS15 15th order Gauss-Radau integrator \citep{reboundias15}
until we can determine the system's state.

\subsubsection{Determining the state}
\label{subsubsec:state}
We integrate the system 5 (third-body) orbits at a time, capturing the positions and velocities of each particle afterwards. \RGM{I'm not sure what the 5 means?}
%Between each step we attempt to assign an end-state to the system.
\RGM{We describe the type of orbit by}
%This is done by 
computing $i$ and $\Omega$ \RGM{include equations and descriptions for i and $\Omega$ here} and checking them against
two criteria:
\begin{enumerate}
    \item Has $\Delta\Omega$ changed sign? In a precessing orbit $\Omega$ changes monotonically. If 
    $\frac{d\Omega}{dt}$ changes from positive to negative (or vise-versa) then we can be sure we have found a librating state.
    \label{item:omega}
    \item Has $i~\sin{\Omega}$ crossed zero? In a librating state this quantity will never change sign. 
    \label{item:sign}
\end{enumerate}
 Determining the state is done iteratively and only requires that $i$ and $\Omega$
be known at each timestep. If neither criteria are true then we do not yet have enough information, and we continue the integration.
If criterion \ref{item:omega} is true then the system is librating. If criterion \ref{item:sign} is true then the system is precessing.
There is no configuration that would allow both criteria to be true.

In order to compute $i$ and $\Omega$ we must first define a coordinate system. {\sc rebound} outputs the positions and
velocities of each particle in the center-of-mass frame. We then use the binary to define a time-dependent coordinate system.
We define the $z$-axis to be along the angular momentum vector of the binary, the $x$-axis to be along the eccentricity vector, and the
$y$-axis so that the coordinate system is right-handed.

\begin{equation}
\begin{aligned}
    \hat{\bm{z}} & = \bm{L}/\|\bm{L}\| \\
    \hat{\bm{x}} & = \bm{e}/\|\bm{e}\| \\
    \hat{\bm{y}} & = - \hat{\bm{x}} \times \hat{\bm{z}}
    \label{eq:coords}
\end{aligned}
\end{equation}
We can then compute $i$ and $\Omega$ from the angular momentum $\bm{L}$.

\begin{equation}
    \cos{i} = \frac{\bm{L} \cdot \hat{z}}{\|\bm{L}\|}
    \label{eq:inclination}
\end{equation}

\begin{equation}
    \Omega = \arctantwo (\bm{L} \cdot \hat{\bm{x}}, -\bm{L} \cdot \hat{\bm{y}})
    \label{eq:omega}
\end{equation}


\begin{figure}
    \begin{centering}
        \includegraphics{figures/sim_example.pdf}
        \caption{Example of system simulation and state determination. This figure tracks the orbital parameters
        of a massless third body orbiting a $1 M_\odot$ binary with equal mass stars and $e_b = 0.4$.
        Initially $\Omega = \frac{\pi}{2}$.
        Solid lines indicate
        the portion of the simulation that is required to determine the state. Dashed lines complete the evolution back to the system's
        initial conditions but are not needed for our analysis. In general, we will only integrate until the state of the system can be
        determined.}
        \label{fig:sim_example}
        \script{sim_example.py}
    \end{centering}
\end{figure}

Figure \ref{fig:sim_example} shows a set of example simulations along with their determined states. \RGM{describe parameters of the system, masses, eccentricity etc} Visualizing the 
system in this way makes it clear why we have chosen our two criteria. Librating orbits will never cross the x-axis
and will cause the polar angle ($\Omega$) to turn back before reaching $0$ or $\pi$. Precessing orbits (both prograde and retrograde)
travel approximately in circles with near-constant $i$.

\section{Results}
\label{sec:results}

\section{Discussion}
\label{sec:discussion}

\section{Conclusions}
\label{sec:conclusions}





\section{Acknowledgements}
\label{sec:ack}

This manuscript was prepared using the open-science software \href{https://show-your.work/en/latest/intro/}{\showyourwork} \citep{luger2021}, making the article completely
reproducible. The source code to compile this document and create the figures is available on GitHub\footnote{\ghurl}.

Simulations in this paper made use of the REBOUND N-body code \citep{rebound}.
The simulations were integrated using IAS15, a 15th order Gauss-Radau integrator \citep{reboundias15}. 

[\textcolor{red}{Cherry Creek}]

\bibliography{cbdisk,syw,reb}

\end{document}
