% Define document class
\documentclass[twocolumn]{aastex631}
\usepackage{showyourwork}
\usepackage{amsmath}
\usepackage{bm}

\newcommand\ghurl[0]{\url{https://github.com/tedjohnson12/bin-disk-paper}}
\newcommand\codeurl[0]{\url{https://github.com/tedjohnson12/misaligned_cb_disk}}
\newcommand\getref[0]{(\textcolor{red}{ref?})}
\newcommand\question[1]{[\textcolor{red}{#1}]}

\newcommand{\TJ}[1]{\textcolor{orange}{#1}}
\newcommand{\RGM}[1]{\textcolor{cyan}{#1}}

\DeclareMathOperator{\arctantwo}{arctan2}
\DeclareMathOperator{\arccsc}{arccsc}

% Begin!
\begin{document}

% Title
\title{Probabilities of polar aligned disks in binary star systems}

% Author list
\author[0000-0002-1570-2203]{Ted Johnson}
%\affiliation{Department of Physics and Astronomy, University of Nevada, Las Vegas, 4505 South Maryland Parkway, Las Vegas, NV 89154, USA}
\author[0000-0003-2401-7168]{Rebecca G. Martin}
%\affiliation{Department of Physics and Astronomy, University of Nevada, Las Vegas, 4505 South Maryland Parkway, Las Vegas, NV 89154, USA}
\author[0000-0003-2270-1310]{Stephen Lepp}
\affiliation{Nevada Center for Astrophysics, University of Nevada, Las Vegas, 4505 South Maryland Parkway, Las Vegas, NV 89154, USA}
\affiliation{Department of Physics and Astronomy, University of Nevada, Las Vegas, 4505 South Maryland Parkway, Las Vegas, NV 89154, USA}
\author[0000-0002-4636-7348]{Stephen H. Lubow}
\affiliation{Space Telescope Science Institute, 3700 San Martin Drive, Baltimore, MD 21218, USA}

% Abstract with filler text
\begin{abstract}
    Circumbinary gas disks that are misaligned to the binary orbital plane evolve toward either a coplanar or a polar-aligned configuration with respect to the binary host.
    The preferred alignment can be found analytically
    in the limit of a massless disk, however disk-binary interactions make the massive disk case much more challenging.
    We employ the N-body code {\sc rebound} to investigate the dynamics of  3-body systems and use Monte Carlo methods
    to quantify the probabilities of each alignment for an initially randomly orientated disk around a given system.
\end{abstract}

\section{Introduction}
\label{sec:intro}



Observed circumbinary planets -- those orbiting around two stars -- have been mostly found to orbit in the same plane as
their hosts \citep{doyle2011,orosz2012,welsh2012}. However, this is likely a result of selection effects \RGM{(citations by David Martin..)}.  Misaligned disks around eccentric binaries evolve \RGM{either to coplanar (citations)} to a polar state  \citep{aly2015,martin2017}. A polar-aligned debris disk has been found in the 99 Her system
\citep{kennedy2012} and polar-aligned gas disks have been found in the HD 98800 \citep{kennedy2019} and V773 Tau systems \citep{kenworthy2022}.
AC Her, a post-AGB system with a polar-inclined
disk, may host the first circumbinary planet found in a polar orbit \citep{hillen2015,anugu2023,martin2023}.


The alignment of circumbinary planets, and the alignment of the disks from which they form, are determined by their interaction history with their
host stars. \RGM{For a massless} circumbinary disk, the type of nodal precession can be split into two categories:

\begin{enumerate}
    \item {\bf Precession about the angular momentum vector of the binary.} In the case where the planet/disk
        is slightly misalignend, its own angular momentum vector precesses about that of the binary \citep[e.g.,][]{bate2000,lubow2000}.
        This is characterized by near constant
        inclination and oscillation of the longitude of the ascending node. Included in this case is the subclassification
        of retrograde orbits (those whose angular momentum vectors are misaligend by $\sim 180 \deg$) and the two
        trivial cases of perfectly aligned/misaligend orbits which do not precess. \label{enum:precession}
    \item {\bf Libration about the eccentricity vector of the binary.} In the case where misalignment is high (i.e. near
    perpendicular) the angular momentum vector of the planet/disk precesses about the eccentricity vector of the binary 
    \citep[e.g.,][]{verrier2009,farago2010,doolin2011}. This is called a librating orbit. In this case the mutual inclination of the disk oscillates about the polar configuration.
    \label{enum:libration}
\end{enumerate}

\RGM{However, for a massive disk, there are additional types of nodal precesion.... (describe) \citep{abod2022}.}

Planets form with the orbital characteristics given by their progenitor disks \citep[e.g.,][]{childs2021}, so it is sufficient in this work to
discuss disk dynamics only.
Differential precession between adjacent annular regions in a disk cause dissipation, ultimately dampening precession or libration
so that the disk reaches a steady state on its viscous timescale \citep[see also \citet{nixon2011,foucart2013,foucart2014}]{bate2000}.
Therefore, it is appropriate to map a disk's
initial interactions with a binary (prograde precession, libration, retrograde precession) to the disk's final configuration
(prograde coplanar, circumpolar, retrograde coplanar, respectively).

In the case of a massless disk, the type of interaction can be determined analytically from just the binary eccentricity and the
direction of the disk's specific angular momentum vector
\citep[parameterized by the disk inclination and its longitude of the ascending node][]{zanazzi2018}.
Given priors on these quantities we can predict the observed frequencies of circumbinary disks (and planets) in the polar configuration.
\citet{ceppi2024} do just this -- finding the fractions of binaries with polar disks and the mean eccentricities of those binaries
as a function of two free parameters.

In this work we will study the more general case of a massive disk. The resulting polar frequency function only contains one
additional parameter, but the orbital parameters of the binary can now change due to interactions with the disk. The additional
complications of this case require the application of numerical simulations; we use the N-body code {\sc rebound} \citep{rebound} to simulate the
dynamics of a three-body system, with a massive point particle used as a stand-in for the disk. \citet{abod2022} found that this 3-body
setup is a valid description of a disk with angular momentum equal to that of the third body in the limit that the disk mass does not change
(i.e. no accretion). This setup also cannot account for warping effects and disk breakage as we must treat the disk as a solid body. 
We describe this setup in detail in Section \ref{sec:methods}

\question{describe the regime we are working in}


\section{Analytic Treatment}
\label{sec:analytic}
For a massive ring that is misaligned to the binary orbit there are two conditions that determine if the ring is librating \citep[see Equations 31, 32, 38 in][]{martin2019}. Each condition is valid for a different regime of the constant of motion $\chi$ that is defined with
\begin{equation}
    \chi = e_{\rm b}^2 - 2\,(1 - e_{\rm b}^2)\,j\,(2j + \cos{i})
\end{equation}
for binary eccentricity $e_{\rm b}$, normalized ring angular momentum $j = J_{\rm r}/J_{\rm b}$, and mutual inclination $i$. The condition for libration is
\begin{equation}
    \label{eq:lib_condition}
        \begin{array}{lr}
            -(1-e_\text{b}^2)(2j+\cos{i})^2 + 5e_\text{b}^2 \sin^2{i}\sin^2{\Omega} > 0 & \chi > 0 \\
            e_\text{b}^2 + 4j(1-e_\text{b}^2)(-\cos{i} + j(-2+5\sin^2{i}\sin^2{\Omega})) > 0 & \chi < 0 .
        \end{array}
\end{equation}
These conditions can be rearranged to find the minimum value of $\Omega$ that will allow for a librating orbit.
\begin{equation}
    \label{eq:omega_min}
    \sin^2{\Omega_{\rm min}} = 
    \left \{
    \begin{array}{lr}
         \frac{(1-e_{\rm b}^2) (2j + \cos^2{i})}{5 e_{\rm b}^2 \sin^2{i}} & \text{if} ~\chi \ge 0 \\
    \frac{2}{5 \sin^2{i}} + \frac{\cos{i}}{5j \sin^2{i}} - \frac{e_{\rm b}^2}{20 j^2 (1-e_{\rm b}^2) \sin^2{i}} & \text{if} ~\chi < 0
    \end{array}
    \right .
\end{equation}

Since the range of $\sin^2{\Omega_{\rm min}}$ is $[0,1]$, there exists a critical value for the inclination, $i_{\rm crit}$, below which all orbits circulate. For the $\chi \ge 0$ case, we require
\begin{equation}
    \label{eq:icrit1}
    \frac{5 e_{\rm b}^2}{1-e_{\rm b}^2} \sin^2{i_{\rm crit}} - (\cos{i_{\rm crit}} + 2j)^2 = 0
\end{equation}
and for the $\chi < 0$ case, $i_{\rm crit}$ satisfies
\begin{equation}
    \label{eq:icrit2}
    \sin^2{i_{\rm crit}} - \frac{\cos{i_{\rm crit}}}{5j} + \frac{e_{\rm b}^2}{20 j^2 (1-e_{\rm b}^2)} + \frac{2}{5} = 0.
\end{equation}
If the left hand side of equation~(\ref{eq:icrit1}) or~(\ref{eq:icrit2}) (depending on the value of $\chi$) evaluates to $<0$, then only circulating orbits are possible.

Given a uniformly distributed $\Omega$, the probability of a polar orbit  is given by
\begin{equation}
    P(j,e_{\rm b}, i) = 
    \left \{
    \begin{array}{lr}
         1 - \frac{2}{\pi} \Omega_{\rm min} & \text{if} ~i > i_{\rm crit} \\
        0 & \text{if} ~i \le i_{\rm crit}
    \end{array}
    \right .
\end{equation}
\citep{zanazzi2018,ceppi2024}.
The fraction of polar orbits is then calculated with
\begin{equation}
    \label{eq:fp_int}
    f_{\rm polar} = \iiint P(j,e_{\rm b}, i)\,p_j(j)\,p_{e_\text{b}}(e_\text{b})\,p_i(i)\, dj\,de_{\rm b}\,di,
\end{equation}
where $p_j(j)$, $p_{e_\text{b}}(e_\text{b})$, and $p_i(i)$ are the probability distributions of $j$, $e_\text{b}$, and $i$, respectively.

This integral is straightforward to compute numerically, and allows for any desired distribution in $j$, $e_{\rm b}$, or $i$, so long as they are normalized. For example, an isotropic distribution of disk \RGM{orientation relative to the binary} would require $p_i(i) = \frac{1}{2}\sin{i}$ for $i$ in the range of $0$ to $\pi$.

\subsection{High-$j$ limit}
\label{subsec:high_j}

When $j$ is very large, the $\chi < 0$ branch of Equation \ref{eq:omega_min} becomes
\begin{equation}
    \sin^2{\Omega_{\rm min}} = \frac{2}{5}\csc^2{i}
\end{equation}
and the value of $i_{\rm crit}$ is 
\begin{equation}
    i_{\rm crit} = \arccsc{\sqrt{5/2}} = 39.2^\circ.
\end{equation}
This is the inclination of the last circular orbit for Kozai-Lidov oscillations in the limit that the perturbed object is very far from the perturber \citep{vonzeipel1910,kozai1962,lidov1962}. Note that it has no dependence on the disk angular momentum or on the binary eccentricity. 

For an isotropic angular momentum distribution,  the polar fraction in this limit is
\begin{equation}
    f_{{\rm polar}, j \gg 1} = \int_0^\pi\,\left (1-\frac{2}{\pi}\arcsin{\left (\sqrt{\frac{2}{5}} \csc{i} \right )}\right ) \sin{i}\,di = \variable{output/high_j_integral.txt}
\end{equation}
This is the value that the higher $j$ curves in Figure \ref{fig:ebin} tend to for low $e_{\rm b}$.

To interpret the Kozai-Lidov oscillations in this limit we can picture a system from the point of view of a far-away stationary observer. When $j \approx 0$, the orbital parameters of the interior binary are not affected by the disk, and the disk undergoes precession or libration due to the effects of the binary. When $j$ is high, however, it is the disk that appears static. An observer would see oscillations in the orbital parameters of the central binary, and its angular momentum and eccentricity vectors would change in response to the disk. This can be thought of in terms of the classical Kozai-Lidov problem, where instead of a planet and its satellite, the interior objects are the binary, and the exterior perturbing object is the circumbinary disk.

%\section{Runge-Kutta-Fehlburg method}
\section{Secular Approximation}
\label{sec:rk}
\citet[Equations (7-10)]{martin2019} also provide a set of coupled first-order differential equations that describe the evolution of $e_{\rm b}$ along with the direction of the disk angular momentum $(\ell_x, \ell_y, \ell_z)$, based on previous work by \citet{farago2010}. We solve these equations using the Runge-Kutta-Fehlberg (RKF) method \TJ{ref?} -- a numerical method similar to the common 4th order Runge-Kutta integrator, but that computes a 5th order solution to allow for a variable step size. This method treats the binary in the quadrupole approximation, meaning that it is not sensitive to the effects of the binary mass fraction or orbital dynamics that occur on timescales less than the binary orbital period \citep[e.g.][]{naoz2016}. However, the computational cost is extremely low; we are able to run $10^4$-$10^5$ simulations per second -- about 10,000 times faster than running $n$-body simulations (see Section \ref{sec:reb}).

We determine the dynamical state of the system by tracking the orbital quantities $x=i\,\cos{\Omega}$ and $y=i\,\sin{\Omega}$, where
\begin{equation}
    i = \arccos{\ell_z}
    \label{eq:inclination}
\end{equation}
and
\begin{equation}
    \Omega = \arctantwo (\ell_x, -\ell_y).
    \label{eq:omega}
\end{equation}
If a disk undergoes prograde nodal precession, then these quantities form clockwise loops about the origin as the system evolves. Under retrograde precession, these loops are counter-clockwise. However, any state that would lead to a polar orbit (i.e. libration and crescent patterns) will form closed loops that do not encompass the origin, and these tracks will never cross the line $i\,\sin{\Omega}=0$. The state can be determined by tracking the quadrant of the point $(i\,\cos{\Omega},~i\,\sin{\Omega})$: prograde orbits will traverse all quadrants in the order [I, IV, III, II], and retrograde orbits will travel in the opposite order. A polar orbit that stars in quadrant I or II, however, is restricted to those two quadrants and will alternate between them. The same goes for polar orbits in quadrants III and IV. Therefore, the dynamical state can often be determined by the list of previous quadrant crossings. For example, I$\rightarrow$IV can only be caused by prograde precession, III$\rightarrow$IV is ambiguous, and III$\rightarrow$IV$\rightarrow$III indicates a polar orbit.

We estimate the value of Equation (\ref{eq:fp_int}) using this method by computing the integral of $P(j,e_{\rm b},i)\,di$ over all $i$; this is the probability of a polar orbit given some $j$ and $e_{\rm b}$. This is done numerically using a grid of simulations, as shown if Figure \ref{fig:grid_example}. These simulations all vary in their initial conditions, with the longitude of the ascending node $\Omega \in [0,2\pi)$ and mutual inclination $i \in [0,\pi]$.

\begin{figure}
    \begin{centering}
        \includegraphics[width=0.5\textwidth]{figures/grid_example.pdf}
    \end{centering}
    \caption{
        Grid of dynamical states for $10^6$ RKF simulations of a system with $e_{\rm b}=0.5$ and $j=0.1$. This grid can be run for any combination of $(e_{\rm b},\,j)$, and integrating over the polar region computes the fraction of systems in which a disk would evolve to a polar state.
    }
    \script{grid_example.py}
    \label{fig:grid_example}
\end{figure}

\section{$n$-body simulations}
\label{sec:reb}

We use the N-body integration software {\sc rebound} \citep{rebound} to simulate a circumbinary system with an additional level of sophistication compared to those described in Section \ref{sec:rk}, in which the binary was treated in the quadrupole approximation. We model the disk as a point mass with its same orbital parameters and angular momentum.

For any given system, the dynamical state is determined similarly to the method described in Section \ref{sec:rk}. Note, however, that the $(i\,\cos{\Omega},~i\,\sin{\Omega})$ tracks now contain small structure on the orbital timescale of the outer mass. These structures can mean that a simulation may not make monotonic progress along its track, occasionally backtracking. Such a backtrack can give a false reading of the dynamical state if it occurs across the $x$ or $y$ axis. To mitigate this problem, we integrate each system until it returns to its original position in $(i\,\cos{\Omega},~i\,\sin{\Omega})$ space. If at, this point, the path does not enclose the origin, then it is polar. Prograde and retrograde motions can be distinguished by the sign of $\langle d\Omega/dt \rangle$. Figure \ref{fig:sim_example} shows paths for a set of simulations initialized with $\Omega = \pi/2$.

\begin{figure}
    \begin{centering}
        \includegraphics[width=0.5\textwidth]{figures/sim_example.pdf}
        \caption{Example of system simulation and state determination. This figure tracks the orbital parameters
        of a third body with $j=0.05$ orbiting a $1 M_\odot$ binary with equal mass stars and $e_b = 0.4$.
        Initially $\Omega = \frac{\pi}{2}$ in all cases.}
        \label{fig:sim_example}
        \script{sim_example.py}
    \end{centering}
\end{figure}

We use the IAS15 Gauss-Radau integrator \citep{reboundias15}, which is a 15th-order variable step size numerical integrator. The maximum timestep this requires is on the order of the binary orbital period, and the integration is therefore much slower than the RKF algorithm, at approximately 2 iterations per second. However, the time required to determine the dynamical state depends on the initial conditions, with some simulation states still undetermined after $10\,000$ outer-mass orbits (~5 seconds). This happens rarely enough that it is most efficient to simply stop the integration, and return an unknown state which must be factored in to our uncertainties.

Given that these simulations take much longer than the RKF algorithm, we cannot compute the same grid of dynamical states. Instead, we employ a Monte Carlo (MC) method to estimate the probability of a polar orbit. For any set of parameters (including the binary fraction $f_{\rm b}$ now) we sample isotropically on the $i$-$\Omega$ sphere, and use the bootstrap method to estimate $f_{\rm polar}$. We sample in batches of 4 and stop when the bootstrap confidence interval metric reaches a prescribed value. After each batch, the results are written to an SQLite database to be recalled in the future. Upon initialization, the MC sampler queries this database for all relevant results.


\begin{figure}
    \begin{centering}
        \includegraphics[width=0.5\textwidth]{figures/compare_massive.pdf}
        \caption{
            MC results (points) for $e_{\rm b}=0.4$, $j=0.67$, and $f_{\rm b}=0.5$ compared to RKF results for the same setup (contour lines). This value of $j$ is chosen here as it is the critical value \citep[see][]{martin2019,abod2022}, beyond which retrograde dynamical states are not allowed.
            Note that some points that {\sc rebound} classifies as polar fall outside the RKF polar region.
            The MC result gives $f_{\text{polar}} = \variable{output/mc_out.txt}$ while the RKF result gives $f_{\text{polar}} = \variable{output/rk_out.txt}$ with
            numerical error on the order of $\variable{output/rk_err.txt}$. \RGM{Explain the black lines}
        }
        \script{compare_massive.py}
        \label{fig:rkf}
    \end{centering}
\end{figure}

Figure \ref{fig:rkf} shows that, while the RKF and {\sc rebound} simulations produce similar results, they are not identical. Specifically noticeable are points sampled by the MC algorithm that lie outside the RKF libration region (with a higher inclination), but are found by {\sc rebound} to librate. There are also points with the opposite discrepancy -- those that lie in the RKF libration region but produce prograde precession -- but they are fewer and not as noticeable by eye. \TJ{(We should try to figure out why this happens)}. However, on the population level, the two measurements of $f_\text{polar}$ lie within the MC uncertainty. As the RKF method is much faster computationally, we will use those results in discussing populations.

\section{Results}
\label{sec:results}

To determine the polar fraction of a given population we must consider three distributions:
\begin{enumerate}
    \item The direction of disk angular momentum $\bm{l}$ \label{it:l} \\
    \item The scalar relative disk angular momentum $j$ \label{it:j} \\
    \item The binary eccentricity $e_{\rm b}$ \label{it:eb}
\end{enumerate}

In Section \ref{sec:analytic} we made the simplifying assumption that $\bm{l}$ is isotropic. Note that this is in contrast to the exponential distribution in
mutual inclination employed by \citet{ceppi2024}. The decaying exponential prefers disks that are aligned with the binary -- valid in the wide-binary regime, but
hydrodynamic simulations by \citet{elsender2023} found that, for binaries with separations on the order of 1 AU, mutual inclinations are seemingly randomly
distributed between $0$ and $\pi$. % See their fig 8

\begin{figure*}[!htbp]
    \begin{centering}
        \includegraphics[width=1\textwidth]{figures/compare_ebin.pdf}
        \caption{
            Polar fraction as a function of $e_{\rm b}$ for various values of $j$. Each fraction is calculated using a $100\times 100$ grid of RKF integrations.
            The dashed lines shows the analytic solution for a massless disk from \citet{zanazzi2018} (in yellow) and from \citet{martin2019} (in gray), integrated numerically.
            There are two competing effects here:
            In the low-$j$ regime the polar fraction is a strong function of $e_{\rm b}$. However, as $j$ increases, $f_{\rm polar}$ is only sensitive to eccentricity
            when $e_{\rm b} \sim 1$, and otherwise prefers a flat distribution.
        }
        \script{compare_ebin.py}
        \label{fig:ebin}
    \end{centering}
\end{figure*}

First, let's not consider the other two distributions, and leave $j$ and $e_{\rm b}$ as free parameters. Figure \ref{fig:ebin} shows that the polar fraction
as a function of $e_{\rm b}$ can be modulated by $j$ to display two behaviors. There exists an analytic solution by \citet{zanazzi2018}, and the tendency is
for $f_{\rm polar}$ to follow this solution either when $j$ is low or $e_{\rm b}$ is very high. Contrarily, if $j$ is sufficiently large, then a flat distribution
(i.e. no dependence on $e_{\rm b}$) is preferred except for when $e_{\rm b}$ is very nearly $1$ (the Kozai-Lidov regime, see Section \ref{subsec:high_j}).

\subsection{An upper limit on the polar fraction}
\label{subsec:fp_limit}
While the true astrophysical distributions of $i$, $e_{\rm b}$, and $j$ are unknown, we can still place constraints on the value of $f_{\rm polar}$.
As mentioned previously, if the disk has no preference for coplanar alignment, then
\begin{equation}
    p_i(i) = \frac{1}{2} \sin{i}
\end{equation}
and we integrate from $0$ to $\pi$.


To decide on a distribution for $e_{\rm b}$ let's first look at the behavior of $\Omega_{\rm min}$ in Equation \ref{eq:omega_min}. Let $\gamma$
be the RHS of that equation. By isolating $\Omega_{\rm min}$ and differentiating we can see that, with respect to $\gamma$,
\begin{equation}
    \frac{d\Omega_{\rm min}}{d \gamma} = \frac{1}{2\sqrt{ \gamma (1 - \gamma)}} > 0
\end{equation}

Differentiating $\gamma$ tells us regardless of $\chi$ that $\frac{d\gamma}{de_{\rm b}} < 0$, so
\begin{equation}
    \frac{d\Omega_{\rm min}}{de_{\rm b}} < 0
\end{equation}
or that the polar fraction always increases with increasing binary eccentricity.


To truly get an upper limit we should choose from a family of distributions one that prefers high eccentricities. However, typical distribution
functions have the form $P(e_{\rm b}) \propto e_{\rm b}^\alpha$ \citep[e.g.][]{hwang2022,ceppi2024}, in which the ``upper limit'' distribution
is one in which $\alpha \rightarrow \infty$ and $P(e_{\rm b})$ approaches the delta function $\delta(e_{\rm b} - 1)$. This would result in a polar frequency
of 1. \citet{hwang2022} suggests that for binaries with separations less than 100 AU it is appropriate to adopt a random distribution (i.e. $\alpha=0$). We then simply
integrate from 0 to 1 with
\begin{equation}
    p_{e_\text{b}}(e_\text{n}) = 1
\end{equation}

We elect not to make any assumptions about the distribution of $j$. We will instead calculate $f_\text{polar}$ for various values of constant $j$. This is analogous by replacing $j$ in equation~(\ref{eq:fp_int}) with $j'$, and letting $p_{j'}(j') = \delta(j'-j)$. We are now interested in the polar fraction at fixed $j$:
\begin{equation}
    f_\text{polar}(j) = \int_{0}^{\pi} \int_{0}^{1} P(j,e_\text{b},i)\, \frac{1}{2} \sin{i}\,de_\text{b}\,di
\end{equation}

Figure \ref{fig:fpol_j} shows this integral computed for a range of $j$. We see that for low $j$, $f_\text{polar} \lessapprox 0.5$.

\begin{figure*}
    \centering
    \includegraphics[width=\textwidth]{figures/fpol_j.pdf}
    \caption{
        Polar fraction $f_\text{polar}$ upper limit as a function of normalized disk angular momentum $j$. These simulations assume a population of binary systems with constant $j$, random $e_\text{b}$, and disks whose angular momentum vectors are distributed isotropically with respect to the binary angular momentum. These values were computed using the analytic technique described in Section \ref{sec:analytic}.
    }
    \script{f_p_of_j.py}
    \label{fig:fpol_j}
\end{figure*}


\section{Discussion}
\label{sec:discussion}

\RGM{Discuss the limitations of the model. e.g.  disk breaking.}

\section{Conclusions}
\label{sec:conclusions}





\begin{acknowledgements}
\label{sec:ack}

We acknowledge support from NASA through grants 80NSSC21K0395 and 80NSSC19K0443. This manuscript was prepared using the open-science software \href{https://show-your.work/en/latest/intro/}{\showyourwork} \citep{luger2021}, making the article completely
reproducible. The source code to compile this document and create the figures is available on GitHub\footnote{\ghurl}.
Simulations in this paper made use of the {\sc rebound} N-body code \citep{rebound}.
The simulations were integrated using IAS15, a 15th order Gauss-Radau integrator \citep{reboundias15}. 

\end{acknowledgements}

\bibliography{cbdisk,syw,reb}

\end{document}
