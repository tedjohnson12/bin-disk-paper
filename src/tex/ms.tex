% Define document class
\documentclass[twocolumn]{aastex631}
\usepackage{showyourwork}

\newcommand\ghurl[0]{\url{https://github.com/tedjohnson12/bin-disk-paper}}
\newcommand\getref[0]{(\textcolor{red}{ref?})}
\newcommand\question[1]{[\textcolor{red}{#1}]}

% Begin!
\begin{document}

% Title
\title{Probabilites of polar aligned disks in binary star systems}

% Author list
\author{Ted Johnson}
\affiliation{Department of Physics and Astronomy, University of Nevada, Las Vegas, 4505 South Maryland Parkway, Las Vegas, NV 89154, USA}
\author{Rebecca Martin}
\affiliation{Department of Physics and Astronomy, University of Nevada, Las Vegas, 4505 South Maryland Parkway, Las Vegas, NV 89154, USA}
\author{Stephen Lepp}
\affiliation{Department of Physics and Astronomy, University of Nevada, Las Vegas, 4505 South Maryland Parkway, Las Vegas, NV 89154, USA}

% Abstract with filler text
\begin{abstract}
    Inclined circumbinary disks settle to coplanar or polar-aligned configurations with respect to the binary host.
    The preferred alignment can be found analytically
    in the case of a massless disk, however disk-star interactions make the general case much more challenging.
    We employ the N-body code REBOUND to investigate the dynamics of these 3-body systems and use monte-carlo methods
    to quantify the probabilities of the various configurations in a given system.
\end{abstract}

\section{Introduction}
\label{sec:intro}



Circumbinary planets -- those orbiting two stars -- have been mostly found to orbit in the same plane as
their hosts \citep{doyle2011,orosz2012,welsh2012}. However, \citet{aly2015} and \citet{martin2017} found that disks around eccentric binaries
can evolve to a polar state if their inclination is perturbed from coplanar. A polar-aligned debris disk has been found in the 99 Her system
\citep{kennedy2012} and polar-aligned gas disks have been found in the HD 98800 and V773 Tau systems \citep[respectively]{kennedy2019,kenworthy2022}.
AC Her, a post-AGB system with a polar-inclined
disk, is believed to host the first circumbinary planet found in a polar orbit \citep{martin2023}.


The orbital parameters of these planets, and the disks that they form from, are determined from their interaction history with their
host stars. These interactions can be broadly split into two catagories:

\begin{enumerate}
    \item {\bf Precession about the angular momentum vector of the binary.} In the case where the planet/disk
        is slightly misaligend, its own angular momentum vector precesses about that of the binary \citep[e.g.,][]{bate2000,lubow2000}.
        This is characterized by near constant
        inclination and oscillation of the longitude of the ascending node. Included in this case is the subclassification
        of retrograde orbits (those whose angular momentum vectors are misaligend by $\sim 180 \deg$) and the two
        trivial cases of perfectly aligned/misaligend orbits which do not precess. \label{enum:precession}
    \item {\bf Libration about the eccentricity vector of the binary.} In the case where misalignment is high (i.e. near
    perpendicular) the angular momentum of the planet/disk will librate about the eccentricity vector of the binary 
    \citep[e.g.,][]{verrier2009,farago2010,doolin2011}.
    \label{enum:libration}
\end{enumerate}

Planets form with orbital characteristics given by their progenitor disks \citep[e.g.,][]{childs2021}, so it is sufficient in this work to
discuss disk dynamics only.
Differential precession between adjacent annular regions in a disk cause dissapation, ultimately dampening precession or libration
so that the disk reaches a steady state on its viscous timescale \citep[see also \citet{nixon2011,foucart2013,foucart2014}]{bate2000}. Therefore, it is appropriate to map a disk's
initial interactions with a binary (prograde precession, libration, retrograde precession) to the disk's final configuration
(prograte coplanar, circumpolar, retrograde coplanar, respectively).

In the case of a massless disk, the type of interaction can be determined analytically from just the binary eccentricity and the
direction of the disk's specific angular momentum vector
\citep[parameterized by the disk inclination and its longitude of the ascending node][]{zanazzi2018}.
Given priors on these quantities we can predict the observed frequencies of circumbinary disks (and planets) in the polar configuration.
\citet{ceppi2024} do just this -- finding the fractions of binaries with polar disks and the mean eccentricities of those binaries
as a function of two free parameters.

In this work we will study the more general case of a massive disk. The resulting polar frequency function only contains one
additional parameter, but the orbital parameters of the binary can now change due to interactions with the disk. The additional
complications of this case require the application of numerical simulations; we use the N-body code REBOUND \citep{rebound} to simulate the
dynamics of a three-body system, with a massive point particle used as a stand-in for the disk. In Section \ref{sec:methods}
we describe this setup in detail.



\section{Methods}
\label{sec:methods}

\section{Results}
\label{sec:results}

\section{Discussion}
\label{sec:discussion}

\section{Conclusions}
\label{sec:conclusions}





\section{Acknowledgements}
\label{sec:ack}

This manuscript was prepared using the open-science software \href{https://show-your.work/en/latest/intro/}{\showyourwork} \citep{luger2021}, making the article completely
reproducible. The source code to compile this document and create the figures is available on GitHub\footnote{\ghurl}.

Simulations in this paper made use of the REBOUND N-body code \citep{rebound}.
The simulations were integrated using IAS15, a 15th order Gauss-Radau integrator \citep{reboundias15}. 

[\textcolor{red}{Cherry Creek}]

\bibliography{cbdisk,syw,reb}

\end{document}
