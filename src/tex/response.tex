\documentclass{letter}
\usepackage{xcolor}
\usepackage{geometry}
\geometry{margin=1in}
\usepackage{amsmath}
\usepackage{bm}
\usepackage{url}

\fboxrule=3pt
\fboxsep=4pt


\newenvironment{response}[1]{
    \fcolorbox{blue}{white}{
    \begin{minipage}{0.46\textwidth}
        #1
    \end{minipage}
    }\hfill
    \begin{minipage}{0.46\textwidth}
    }
    {\end{minipage}}


\begin{document}

\begin{letter}{}
    \opening{Dear Editor Carraro, Data Editors, and Anonymous Reviewer,}

    Please find our revised manuscript alongside the below response to the reviewers report. The authors would like to thank the editor, reviewer, and data editors for their thoughtful review, which has lead to a greatly improved manuscript. Changes have been highlighted in \textcolor{blue}{\bf blue}.

    \begin{response}{Reviewer's Comment}
        Authors' Response
    \end{response}

    \begin{response}
        {
            L35-38: The evidence pointed out by the authors for AC Her is extremely indirect and also around an evolved binary star with a CBD, for which the disc formation pathway and subsequent evolution remains very unclear. At the moment, this "evidence" is loosely supported by data as the planet was not confirmed/observed but simply proposed as the "most likely scenario" (Martin et al., 2023) based on disc truncation arguments alone. Therefore, I would strongly suggest to rephrase this part to avoid conveying the wrong idea to non-specialists.
        }
        We have changed the order of the text and changed the language so that this portion is clearer.
    \end{response}

    \begin{response}
        {L93: A bit too self-referential in my opinion. SPH simulations on this very topic were also done by other teams (Aly+2015; Cuello+2019, Young+2023, Ceppi+2023). It is not necessary to be exhaustive here, but I would avoid citing their own work exclusively.}
        Thank you for these suggestions. We have added the references.
    \end{response}

    \begin{response}
        {L128: "Code implementing" typo? If so, please rephrase.}
        This is not a typo, but has been rephrased for clarity.
    \end{response}

    \begin{response}
        {L142: Equation 2, I believe two "if" statements are missing.}
        These statements have been added.
    \end{response}

    \begin{response}
        {L184: Perhaps specify that "isotropic angular momentum distribution" means that the relative angle between the disc and binary orbital planes is uniformly distributed. This would help to give a more clear astrophysical interpretation.}
        By ``isotropic'' we mean that the specific angular momentum vector $\bm{\ell}$ has no preference towards any infinitesimal solid angle $d\Omega$. We have added text to clarify this.
    \end{response}

    \begin{response}
        {L239: I would suggest defining or explaining how quadrants are displayed. I assume the anti-clockwise convention is used here.}
        This is correct. We have added some text to specify this convention.
    \end{response}

    \begin{response}
        {Fig. 6: Please comment the insert.}
        We have added a more descriptive caption in addition to a description of the insert.
    \end{response}

    \begin{response}
        {Fig. 7: the caption could be more descriptive to guide the reader.}
        We have added a more descriptive caption.
    \end{response}

    \begin{response}
        {L426: drops instead of drop?}
        This has been fixed. Thank you for catching it.
    \end{response}

    \begin{response}
        {L445: In the definition of f, is M meant to be M\_b or M\_1 + M\_2? It is likely the latter one. Unless I am mistaken, it was not defined earlier.}
        It is supposed to be $M_\text{b}\equiv M_1 + M_2$. We have added text for clarification.
    \end{response}

    \begin{response}
        {L468: Again, I think m\_r was not defined earlier. I assume this is the mass of the ring. Also, would it be possible to write down the expression of Gamma\_2 to effectively see that it tends to 1 when j is large? Since it is very complicated it could go in the Appendix to avoid interrupting the flow of the section.}
        We have added text to define $m_\text{r}$. We also added a section in the appendix to describe the process of computing $\Gamma_2$. However, we also clarified in the text why $\Gamma_2$ tends to 1 for small $j$, so this section may no longer be necessary.
    \end{response}

    \begin{response}
        {L473: Syntax/repetition issue. Please rephrase.}
        We have fixed the issue by clarifying that we are translating the precession timescale from code units to physical units.
    \end{response}

    \begin{response}
        {Fig. 8: The bottom part of this figure could be further commented in Sect. 6.1.}
        We have added text in Section 6.1 discussing this panel.
    \end{response}

    \begin{response}{
        L532: Bohn+2022 could be a relevant reference to add here.
    }
     Thank you for the great suggestion. We have added the reference.   
    \end{response}

    \begin{response}
        {L580: I would conclude the discussion (or add to the conclusion) that, besides imposing constraints on the disc radial extend and j, the two studied mechanisms (disc breaking and self-gravity) are expected to lower the fraction of polar CBDs. If I understood correctly, large values of j and radially extended discs are unlikely to become polar (as a whole at least). Therefore, the statistical relevance/weight of this subgroup of discs should be less important at the overall polar CBD population level. This could be briefly commented at the end.}

        This is a great suggestion, however the overall effect is more complicated. For disks that break, $f_\text{polar}$ being ``the fraction of disks that evolve to a polar configuration'' may be less relevant than ``the fraction of systems that host a polar disk''. The main effect of disk breaking is that the alignment of the inner disk happens more quickly, reducing the fraction of systems which have neither a polar nor coplanar disk. The main effect of gravitational instability is that massive disks must be more radially extended, which could possibly lead to warps and breaks. The total effect likely would be that a single extended disk evolves into multiple disks with various degrees of alignment, whether that be polar or coplanar. We have added text at the end of the discussion to highlight these effects.
    \end{response}

    \begin{response}{
        L615: Unclear what "1/s" means. Below a second per calculation/timestep? Please specify.
    }
        We have clarified that this refers to the time to run a single simulation until it can be determined if the third body is undergoing circular precession or libration.
    \end{response}

    \begin{response}{
        L628: Sections (missing s).
    }
    This has been fixed.
    \end{response}

    \begin{response}{
        L642: At the end of the conclusions (or else at the end of the discussion), a brief discussion on how future observations could test these findings would add value to the article.
    }
        We have added a discussion about the possibility of confirming this work though observations. We propose a survey to measure the inclinations of disks in young eclipsing binary systems could reasonably constrain the polar fraction.
    \end{response}

    \begin{response}{
        Data Editor's Comment
    }
    Author's Response
    \end{response}

    \begin{response}
        {I. Is this a showyourwork paper? If it is then we might be missing files necessary to properly augment the final article with code links. And we would need to add a paragraph to the text explaining the showyourwork icons. Would you contact the Data Editors helpdesk if you need assistance ensuring that the showyourwork elements show up in the final article. }

    \end{response}

    \begin{response}
        {
            II. I don't see the Zenodo DOI results tagged in the text. Can that be added using the instructions here: \url{https://journals.aas.org/aastexguide/#datasets}
        }
    \end{response}

    \begin{response}
        {
            III. For Showyourwork papers, we would like to you to freeze the final final version on GitHub and add that Zenodo DOI to the text as well. More information is here: \url{https://journals.aas.org/aastexguide/#datasets}

            [2] \url{https://journals.aas.org/data-guide/#repositories}

[3] \url{https://docs.github.com/en/repositories/archiving-a-github-repository/referencing-and-citing-content} 
        }
    \end{response}

    \begin{response}
        {
            IV. It appears you intend tedjohnson12/polar-disk-freq to be reusable software. If so we ask that you again freeze a copy of the final final version and cite that frozen copy with the final article.

[1] \url{https://journals.aas.org/policy-statement-on-software/}

[2] \url{https://journals.aas.org/data-guide/#repositories}

[3] \url{https://docs.github.com/en/repositories/archiving-a-github-repository/referencing-and-citing-content}
        }
        We have released this software as \texttt{v1.0.1} and cited the Zenodo DOI in the text.
    \end{response}


\end{letter}

\end{document}
