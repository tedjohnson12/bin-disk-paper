\documentclass{letter}
\usepackage{xcolor}
\usepackage{geometry}
\geometry{margin=1in}
\usepackage{amsmath}

\fboxrule=3pt
\fboxsep=4pt


\newenvironment{response}[1]{
    \fcolorbox{blue}{white}{
    \begin{minipage}{0.46\textwidth}
        #1
    \end{minipage}
    }\hfill
    \begin{minipage}{0.46\textwidth}
    }
    {\end{minipage}}


\begin{document}

\begin{letter}{}
    \opening{Dear Editor Carraro, Data Editors, and Anonymous Reviewer,}

    Please find our revised manuscript alongside the below response to the reviewers report. The authors would like to thank the editor, reviewer, and data editors for their thoughtful review, which has lead to a greatly improved manuscript.

    \begin{response}{Reviewer's Comment}
        Authors' Response
    \end{response}

    \begin{response}
        {Fig. 7: the caption could be more descriptive to guide the reader.}
        We have added a more descriptive caption.
    \end{response}

    \begin{response}
        {L426: drops instead of drop?}
        This has been fixed. Thank you for catching it.
    \end{response}

    \begin{response}
        {L445: In the definition of f, is M meant to be M\_b or M\_1 + M\_2? It is likely the latter one. Unless I am mistaken, it was not defined earlier.}
        It is supposed to be $M_\text{b}\equiv M_1 + M_2$. We have added text for clarification.
    \end{response}

    \begin{response}
        {L468: Again, I think m\_r was not defined earlier. I assume this is the mass of the ring. Also, would it be possible to write down the expression of Gamma\_2 to effectively see that it tends to 1 when j is large? Since it is very complicated it could go in the Appendix to avoid interrupting the flow of the section.}
        We have added text to define $m_\text{r}$. We also added a section in the appendix to describe the process of computing $\Gamma_2$. However, we also clarified in the text why $\Gamma_2$ tends to 1 for small $j$, so this section may no longer be necessary.
    \end{response}

    \begin{response}
        {L473: Syntax/repetition issue. Please rephrase.}
        We have fixed the issue by clarifying that we are translating the precession timescale from code units to physical units.
    \end{response}

    \begin{response}
        {Fig. 8: The bottom part of this figure could be further commented in Sect. 6.1.}
        We have added text in Section 6.1 discussing this panel.
    \end{response}

    \begin{response}{
        L532: Bohn+2022 could be a relevant reference to add here.
    }
     Thank you for the great suggestion. We have added the reference.   
    \end{response}

    \begin{response}
        {L580: I would conclude the discussion (or add to the conclusion) that, besides imposing constraints on the disc radial extend and j, the two studied mechanisms (disc breaking and self-gravity) are expected to lower the fraction of polar CBDs. If I understood correctly, large values of j and radially extended discs are unlikely to become polar (as a whole at least). Therefore, the statistical relevance/weight of this subgroup of discs should be less important at the overall polar CBD population level. This could be briefly commented at the end.}
        This is a great suggestion, however I think the overall effect is a bit different. High-$j$ disks have polar probabilities that are sensitive to initial inclination and insensitive to $e_{\rm b}$. Radially extended disks have long alignment timescales and may not evolve much within their lifetimes. Reducing these two populations (which should have significant overlap) would make $f_\text{polar}$ more sensitive to $e_{\rm b}$ and reduce the population of disks that end their lives outside of polar or coplanar alignments.
    \end{response}

    \begin{response}{
        L615: Unclear what "1/s" means. Below a second per calculation/timestep? Please specify.
    }
        We have clarified that this refers to the time to run a single simulation until it can be determined if the third body is undergoing circular precession or libration.
    \end{response}

    \begin{response}{
        L628: Sections (missing s).
    }
    This has been fixed.
    \end{response}

    \begin{response}{
        L642: At the end of the conclusions (or else at the end of the discussion), a brief discussion on how future observations could test these findings would add value to the article.
    }
        We have added a discussion about the possibility of confirming this work though observations. We propose a survey to measure the inclinations of disks in young eclipsing binary systems could reasonably constrain the polar fraction.
    \end{response}




\end{letter}

\end{document}
